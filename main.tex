%!TEX root = interventions.tex
\section{Interventions}
\subsection{Definition}
\begin{frame}{Interventionen}
    \begin{block}{Interventionsverteilung}
        Sei $\mathbb{P}^\mathbf{X}$ die zu einer SEM
        $\mathcal{S} := (\mathcal{S}, \mathbb{P}^N)$ gehörende Verteilung. Dann
        kann eine (oder mehr) Strukturgleichungen aus $\mathcal{S}$ entfernt
        werden ohne einen Zyklus im Graphen zu erzeugen. Die Verteilung des
        neuen SEM $\tilde{\mathcal{S}}$ heißt dann
        \textit{Interventionsverteilung}.

        Bei den Variablen, deren Strukturgleichungen ersetzt wurden, sagt man,
        wurde \textit{interveniert}.

        Die neue Verteilung wird mit
        \[\mathbb{P}_{\tilde{\mathcal{S}}}^{\mathbf{X}} = \mathbb{P}_{\mathcal{S}}^{\mathbf{X}| do(X_j=\tilde{f}(\tilde{\mathbf{PA}}_j, \tilde{N}_j))}\]
        beschrieben.

        Die Menge der Rauschvariablen in $\mathcal{S}$ beinhaltet nun einige
        \enquote{neue} und einige \enquote{alte} $N$'s. $\mathcal{S}$ muss
        paarweise unabhängig sein.
    \end{block}
\end{frame}

\begin{frame}{Nierensteine}
\begin{columns}
    \begin{column}{0.45\textwidth}
        \begin{center}\textbf{Modell A}\end{center}
    \end{column}
    \begin{column}{0.45\textwidth}
        \begin{center}\textbf{Modell B}\end{center}
    \end{column}
\end{columns}
\end{frame}