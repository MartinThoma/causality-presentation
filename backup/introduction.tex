%!TEX root = interventions.tex
\section{SEMs}
\subsection{SEMs}
\begin{frame}{SEMs}
    \begin{block}{Structural Equaltion Model (kurz: SEM)}
        Ein \textit{Struckturgleichungsmodel} ist ein Tupel
        $\mathcal{S} := (\mathcal{S}, \mathbb{P}^\mathbf{N})$, wobei
        $\mathcal{S} = (S_1, \dots, S_p)$ ein Tupel aus $p$ Gleichungen
        \[S_j : X_j = f_j(\mathbf{PA}_j, N_j), \;\;\; j=1, \dots, p\]
        ist und $\mathbf{PA}_j \subseteq \Set{X_1, \dots, X_p} \setminus \Set{X_j}$
        die \textit{Eltern von $X_j$} und $\mathbb{P}^\mathbf{N} = \mathbb{P}^{N_1, \dots, N_p}$
        die gemeinsame Verteilung der Rauschvariablen ist. Diese müssen
        von einander unabhängig sein, $\mathbb{P}^\mathbf{N}$ muss also eine
        Produktverteilung sein.
    \end{block}
\end{frame}
